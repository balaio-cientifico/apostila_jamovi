\documentclass[12pt, oneside]{book}

% Pacotes necessários
\usepackage[portuguese]{babel}
\usepackage[utf8]{inputenc}
\usepackage[scale=1.5]{ccicons}
\usepackage{graphicx}
\usepackage{titlesec}
\usepackage{setspace}
\usepackage{geometry}
\usepackage{indentfirst}
\usepackage{hyperref}
\usepackage{float}
\usepackage{tikz}
\usepackage{kpfonts}
\usepackage{fontawesome5}
\usepackage{fancyhdr}
\usepackage{subfigure}
\usepackage{tcolorbox}
\usepackage{xcolor}
\usepackage{pgfplots}
\usepackage{pgfmath}
\usepackage[style=abnt]{biblatex}

\addbibresource{bibliografia.bib}

\usetikzlibrary{intersections}

% Define o estilo da página
\pagestyle{fancy}
\fancyhf{}
\fancyfoot[L]{Erick Faria}
\fancyfoot[C]{\thepage}
\fancyfoot[R]{\href{https://www.balaiocientifico.com/}{Balaio Científico}}
\fancyhead[R]{\nouppercase{\leftmark}}
\renewcommand{\headrulewidth}{0.4pt}
\renewcommand{\footrulewidth}{0.4pt}
% Redefine o comando \chaptermark para remover o prefixo "Chapter X."
\renewcommand{\chaptermark}[1]{\markboth{#1}{}}


\usefont{T1}{jkpss}{m}{n}

\geometry{
  left=3cm,
  right=2cm,
  top=3cm,
  bottom=2cm,
  bindingoffset=0cm
}

\onehalfspacing

% Define o caminho das imagens
\graphicspath{{imagens/}}

% Define o estilo de início do capítulo
\titleformat{\chapter}[display]
  {\normalfont\huge\bfseries\filcenter}
  {\begin{tikzpicture}[remember picture,overlay]
    \node[anchor=north west,inner sep=0pt] at (current page.north west)
      {\includegraphics[width=\paperwidth,height=0.2\paperheight]{imagens/fundo_cap.pdf}};
    \fill[white,opacity=0] (current page.north west) rectangle (current page.south east);
    \draw[anchor=north west] (0.2\paperwidth,0.2\paperheight) node [font=\Huge\bfseries] {\chaptertitlename\ \thechapter};
  \end{tikzpicture}}
  {0pt}
  {\Huge}

% Define os metadados do documento
\hypersetup{
colorlinks=true,
linkcolor=blue,
filecolor=blue,
citecolor=blue,      
urlcolor=blue,
pdftitle={Apostila de Jamovi},
pdfauthor={Erick Faria},
pdfsubject={Apostila de Jamovi},
pdfkeywords={Jamovi, Estatística, Análise de Dados}
}

\begin{document}
\let\cleardoublepage\clearpage

% Inclui a capa
\begin{titlepage}
	\centering % Center everything on the title page
	\scshape % Use small caps for all text on the title page
	\vspace*{1.5\baselineskip} % White space at the top of the page
% ===================
%	Title Section 	
% ===================

	\rule{13cm}{1.6pt}\vspace*{-\baselineskip}\vspace*{2pt} % Thick horizontal rule
	\rule{13cm}{0.4pt} % Thin horizontal rule
	
		\vspace{0.75\baselineskip} % Whitespace above the title
% ========== Title ===============	
	{	\Huge Introdução ao\\ 
			\vspace{4mm}
		JAMOVI \\	}
% ======================================
		\vspace{0.75\baselineskip} % Whitespace below the title
	\rule{13cm}{0.4pt}\vspace*{-\baselineskip}\vspace{3.2pt} % Thin horizontal rule
	\rule{13cm}{1.6pt} % Thick horizontal rule
	
		\vspace{1.75\baselineskip} % Whitespace after the title block
% =================
%	Information	
% =================
	{\large Produzido por: \href{https://www.balaiocientifico.com/author/erickfaria/}{Erick Faria} \\
		\vspace*{1.2\baselineskip}}
	% \href{mailto:erickfaria@balaiocientifico.com}{erickfaria@balaiocientifico.com}} \\
	\vfill
\url{www.balaiocientifico.com}\\
\vspace{0.5cm}
\ccbyncsa
\end{titlepage}

% Define a numeração das páginas em romanos
\pagenumbering{roman}
\setcounter{page}{1}

% Inclui a página de atribuição

Fico feliz e agradeço por você ter escolhido ler a Apostila do Jamovi. Espero que este recurso seja útil para você em sua jornada de aprendizado e aplicação de análises estatísticas com o Jamovi.

Gostaria de ressaltar que este documento está licenciado sob a licença Creative Commons, o que significa que você tem a liberdade de usar, compartilhar e adaptar o material. No entanto, é importante fornecer as devidas atribuições ao autor original. Essas atribuições são uma maneira de reconhecer e valorizar o trabalho que foi dedicado para criar esta apostila.

É importante mencionar também que a apostila está em constante atualização. À medida que novas versões forem lançadas, novos recursos, exemplos e aprimoramentos serão adicionados. Recomendamos que você visite regularmente o site do \href{https://www.balaiocientifico.com/jamovi/apostila-de-jamovi/} {Balaio Científico} para obter a versão mais atualizada da apostila. Dessa forma, você poderá se beneficiar de novos conteúdos e melhorias à medida que são disponibilizados.

Além disso, incentivo você a compartilhar e usar este material com outras pessoas interessadas em aprender sobre análise estatística com o Jamovi. Acredito no poder do conhecimento compartilhado e na colaboração para promover um aprendizado mais amplo e significativo.

Se você deseja colaborar com esse projeto, você pode acessar o repositório dessa apostila no \href{https://github.com/balaio-cientifico/apostila_jamovi}{Git Hub}. Se quiser entrar em contato comigo você pode me enviar um e-mail: \href{mailto:erickfaria@balaiocientifico.com}{erickfaria@balaiocientifico.com}

Mais uma vez, agradeço sua leitura e interesse pela Apostila do Jamovi. Espero que ela seja útil para você em seus estudos e projetos. Se você tiver alguma dúvida, sugestão ou feedback, não hesite em entrar em contato comigo. Estou sempre buscando melhorar e fornecer recursos de alta qualidade para a comunidade.

Aproveite o material e bons estudos!

\vfill
\small{\noindent \textbf{Você tem o direito de:} \vspace{-3mm}\\
\noindent \rule{3.3cm}{0.5pt} \\
\textbf{Compartilhar } — copiar e redistribuir o material em qualquer suporte ou formato \\
\textbf{Adaptar} — remixar, transformar, e criar a partir do material \\
\\
\small{\noindent \textbf{De acordo com os termos seguintes:} \vspace{-3mm}\\
\noindent \rule{3.3cm}{0.5pt} \\
\textbf{Atribuição} — Você deve dar o crédito apropriado, prover um link para a licença e indicar se mudanças foram feitas. Você deve fazê-lo em qualquer circunstância razoável, mas de nenhuma maneira que sugira que o licenciante apoia você ou o seu uso.  \\
\textbf{NãoComercial} — Você não pode usar o material para fins comerciais.\\
\textbf{CompartilhaIgual} - Se você remixar, transformar, ou criar a partir do material, tem de distribuir as suas contribuições sob a mesma licença que o original.  \\
\\
\centerline{\href{https://creativecommons.org/licenses/by-nc-sa/4.0/deed.pt_BR}{Atribuição-NãoComercial-CompartilhaIgual 4.0 Internacional (CC BY-NC-SA 4.0)}}\\
\centerline{\ccbyncsa}

% Define o nome do sumário
\renewcommand{\contentsname}{Sumário}

% Inclui o sumário
\tableofcontents

% Define a numeração das páginas em arábicos
\pagenumbering{arabic}
\setcounter{page}{1}

% Inclui o primeiro capítulo
\chapter{Introdução ao Jamovi}

\section{O que é o Jamovi?}

O jamovi é um software estatístico com interface gráfica. É um software relativamente novo, se comparado com os seus concorrentes como o SPSS, SAS e PSPP. Com o jamovi você consegue ter um ambiente de fácil aprendizado e fácil manuseio, pois é possível integrar a facilidade do ambiente gráfico, com o poder da linguagem R para a automação de trabalhos.

Além das análises estatísticas convencionais, tais como: estatística descritiva, tabelas cruzadas, boxplot, etc. É possível desenvolver modelos matemáticos. Assim, podemos definir o jamovi como uma solução completa para análises quantitativas e qualitativas com o uso da matemática e estatística. Um software estatístico completo e gratuito.
\chapter{Estatística Descritiva}

A estatística descritiva no Jamovi é uma técnica utilizada para organizar, resumir e apresentar dados de forma clara e concisa. Ela é utilizada para descrever as características dos dados, como a média, mediana, moda, desvio padrão, frequência, entre outros.

No Jamovi, é possível acessar várias ferramentas para realizar estatística descritiva, como gráficos, tabelas e estatísticas básicas. Ele também permite que você crie tabelas cruzadas, boxplots e histogramas para visualizar os dados de forma mais fácil. Além disso, o Jamovi também permite a criação de relatórios de estatísticas descritivas para compartilhar com outros pesquisadores.

\subsection{Média}



$\bar{x} = \frac{1}{n} \sum_{i=1}^{n} x_i$


\subsection{Moda}

$\text{moda} = \text{valor mais frequente na amostra}$

\subsection{Desvio Padrão}

$s = \sqrt{\frac{1}{n-1} \sum_{i=1}^{n} (x_i - \bar{x})^2}$

\chapter{Testes de Hipóteses}
\chapter{Análise de Variância (ANOVA)}
\chapter{Análise de Variância (ANOVA)}

A Análise de Variância (ANOVA) é uma técnica estatística utilizada para comparar as médias de três ou mais grupos independentes. Ela permite determinar se as diferenças observadas entre as médias dos grupos são estatisticamente significativas ou se podem ser atribuídas ao acaso \parencite{Triola2017}.

A ANOVA baseia-se na decomposição da variabilidade total dos dados em duas componentes: a variabilidade entre os grupos e a variabilidade dentro dos grupos. A ideia fundamental por trás da ANOVA é que, se as médias dos grupos são iguais e as variabilidades dentro dos grupos são semelhantes, então qualquer diferença observada entre as médias pode ser atribuída ao acaso. Por outro lado, se as médias dos grupos são diferentes e/ou as variabilidades dentro dos grupos são grandes em relação à variabilidade entre os grupos, então é mais provável que as diferenças observadas sejam estatisticamente significativas.

Na ANOVA, a hipótese nula assume que não há diferenças entre as médias dos grupos, enquanto a hipótese alternativa considera que pelo menos uma das médias é diferente das demais. O teste estatístico da ANOVA utiliza a estatística F, que compara a variabilidade entre os grupos com a variabilidade dentro dos grupos. Se a estatística F é grande o suficiente para rejeitar a hipótese nula, isso indica que pelo menos uma das médias é estatisticamente diferente das demais.

Existem diferentes tipos de ANOVA, dependendo do design do estudo e do número de fatores considerados. A ANOVA de um fator compara as médias de três ou mais grupos independentes, enquanto a ANOVA de dois fatores analisa o efeito de duas variáveis independentes nos grupos. Além disso, a ANOVA pode ser aplicada em diferentes contextos, como experimentos controlados, estudos observacionais ou análise de dados de pesquisas.

A ANOVA é amplamente utilizada em diversas áreas, como ciências sociais, ciências da saúde, ciências naturais e engenharia. Ela fornece uma abordagem estatística robusta para a comparação de médias de vários grupos, permitindo que os pesquisadores determinem se as diferenças observadas são estatisticamente significativas. A ANOVA também permite realizar análises post hoc para identificar quais grupos diferem significativamente entre si, contribuindo para uma compreensão mais aprofundada dos padrões e relações presentes nos dados.
\chapter{Análise de Dados Categóricos}

\printbibliography

% Define o final do documento
\backmatter

\end{document}