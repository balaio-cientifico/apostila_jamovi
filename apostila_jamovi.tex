\documentclass{book}
\usepackage[portuguese]{babel}
\usepackage{graphicx}
\usepackage{titlesec}
\usepackage{hyperref}
\usepackage{tikz}
\usepackage[scale=1.5]{ccicons}
\graphicspath{{imagens/}}

% Define o estilo de início do capítulo
\titleformat{\chapter}[display]
  {\normalfont\huge\bfseries\filcenter}
  {\begin{tikzpicture}[remember picture,overlay]
    \node[anchor=north west,inner sep=0pt] at (current page.north west)
      {\includegraphics[width=\paperwidth,height=0.2\paperheight]{imagens/fundo_cap.pdf}};
    \fill[white,opacity=0] (current page.north west) rectangle (current page.south east);
    \draw[anchor=north west] (0.2\paperwidth,0.2\paperheight) node [font=\Huge\bfseries] {\chaptertitlename\ \thechapter};
  \end{tikzpicture}}
  {0pt}
  {\Huge}

  % Define os metadados do documento
\hypersetup{
colorlinks=true,
linkcolor=blue,
filecolor=magenta,      
urlcolor=blue,
pdftitle={Apostila de Jamovi},
pdfauthor={Erick Faria},
pdfsubject={Apostila de Jamovi},
pdfkeywords={Jamovi, Estatística, Análise de Dados}
}

\begin{document}

\begin{titlepage}
	\centering % Center everything on the title page
	\scshape % Use small caps for all text on the title page
	\vspace*{1.5\baselineskip} % White space at the top of the page
% ===================
%	Title Section 	
% ===================

	\rule{13cm}{1.6pt}\vspace*{-\baselineskip}\vspace*{2pt} % Thick horizontal rule
	\rule{13cm}{0.4pt} % Thin horizontal rule
	
		\vspace{0.75\baselineskip} % Whitespace above the title
% ========== Title ===============	
	{	\Huge Introdução ao\\ 
			\vspace{4mm}
		JAMOVI \\	}
% ======================================
		\vspace{0.75\baselineskip} % Whitespace below the title
	\rule{13cm}{0.4pt}\vspace*{-\baselineskip}\vspace{3.2pt} % Thin horizontal rule
	\rule{13cm}{1.6pt} % Thick horizontal rule
	
		\vspace{1.75\baselineskip} % Whitespace after the title block
% =================
%	Information	
% =================
	{\large Produzido por: \href{https://www.balaiocientifico.com/author/erickfaria/}{Erick Faria} \\
		\vspace*{1.2\baselineskip}}
	% \href{mailto:erickfaria@balaiocientifico.com}{erickfaria@balaiocientifico.com}} \\
	\vfill
\url{www.balaiocientifico.com}\\
\vspace{0.5cm}
\ccbyncsa
\end{titlepage}
\cleardoublepage

Fico feliz e agradeço por você ter escolhido ler a Apostila do Jamovi. Espero que este recurso seja útil para você em sua jornada de aprendizado e aplicação de análises estatísticas com o Jamovi.

Gostaria de ressaltar que este documento está licenciado sob a licença Creative Commons, o que significa que você tem a liberdade de usar, compartilhar e adaptar o material. No entanto, é importante fornecer as devidas atribuições ao autor original. Essas atribuições são uma maneira de reconhecer e valorizar o trabalho que foi dedicado para criar esta apostila.

É importante mencionar também que a apostila está em constante atualização. À medida que novas versões forem lançadas, novos recursos, exemplos e aprimoramentos serão adicionados. Recomendamos que você visite regularmente o site do \href{https://www.balaiocientifico.com/jamovi/apostila-de-jamovi/} {Balaio Científico} para obter a versão mais atualizada da apostila. Dessa forma, você poderá se beneficiar de novos conteúdos e melhorias à medida que são disponibilizados.

Além disso, incentivo você a compartilhar e usar este material com outras pessoas interessadas em aprender sobre análise estatística com o Jamovi. Acredito no poder do conhecimento compartilhado e na colaboração para promover um aprendizado mais amplo e significativo.

Se você deseja colaborar com esse projeto, você pode acessar o repositório dessa apostila no \href{https://github.com/balaio-cientifico/apostila_jamovi}{Git Hub}. Se quiser entrar em contato comigo você pode me enviar um e-mail: \href{mailto:erickfaria@balaiocientifico.com}{erickfaria@balaiocientifico.com}

Mais uma vez, agradeço sua leitura e interesse pela Apostila do Jamovi. Espero que ela seja útil para você em seus estudos e projetos. Se você tiver alguma dúvida, sugestão ou feedback, não hesite em entrar em contato comigo. Estou sempre buscando melhorar e fornecer recursos de alta qualidade para a comunidade.

Aproveite o material e bons estudos!

\vfill
\small{\noindent \textbf{Você tem o direito de:} \vspace{-3mm}\\
\noindent \rule{3.3cm}{0.5pt} \\
\textbf{Compartilhar } — copiar e redistribuir o material em qualquer suporte ou formato \\
\textbf{Adaptar} — remixar, transformar, e criar a partir do material \\
\\
\small{\noindent \textbf{De acordo com os termos seguintes:} \vspace{-3mm}\\
\noindent \rule{3.3cm}{0.5pt} \\
\textbf{Atribuição} — Você deve dar o crédito apropriado, prover um link para a licença e indicar se mudanças foram feitas. Você deve fazê-lo em qualquer circunstância razoável, mas de nenhuma maneira que sugira que o licenciante apoia você ou o seu uso.  \\
\textbf{NãoComercial} — Você não pode usar o material para fins comerciais.\\
\textbf{CompartilhaIgual} - Se você remixar, transformar, ou criar a partir do material, tem de distribuir as suas contribuições sob a mesma licença que o original.  \\
\\
\centerline{\href{https://creativecommons.org/licenses/by-nc-sa/4.0/deed.pt_BR}{Atribuição-NãoComercial-CompartilhaIgual 4.0 Internacional (CC BY-NC-SA 4.0)}}\\
\centerline{\ccbyncsa}

\tableofcontents


\mainmatter

\chapter{Introdução ao Jamovi}

\section{O que é o Jamovi?}

O jamovi é um software estatístico com interface gráfica. É um software relativamente novo, se comparado com os seus concorrentes como o SPSS, SAS e PSPP. Com o jamovi você consegue ter um ambiente de fácil aprendizado e fácil manuseio, pois é possível integrar a facilidade do ambiente gráfico, com o poder da linguagem R para a automação de trabalhos.

Além das análises estatísticas convencionais, tais como: estatística descritiva, tabelas cruzadas, boxplot, etc. É possível desenvolver modelos matemáticos. Assim, podemos definir o jamovi como uma solução completa para análises quantitativas e qualitativas com o uso da matemática e estatística. Um software estatístico completo e gratuito.
% \chapter{Estatística Descritiva}

A estatística descritiva no Jamovi é uma técnica utilizada para organizar, resumir e apresentar dados de forma clara e concisa. Ela é utilizada para descrever as características dos dados, como a média, mediana, moda, desvio padrão, frequência, entre outros.

No Jamovi, é possível acessar várias ferramentas para realizar estatística descritiva, como gráficos, tabelas e estatísticas básicas. Ele também permite que você crie tabelas cruzadas, boxplots e histogramas para visualizar os dados de forma mais fácil. Além disso, o Jamovi também permite a criação de relatórios de estatísticas descritivas para compartilhar com outros pesquisadores.

\subsection{Média}



$\bar{x} = \frac{1}{n} \sum_{i=1}^{n} x_i$


\subsection{Moda}

$\text{moda} = \text{valor mais frequente na amostra}$

\subsection{Desvio Padrão}

$s = \sqrt{\frac{1}{n-1} \sum_{i=1}^{n} (x_i - \bar{x})^2}$


\backmatter
% bibliography, index, etc.

\end{document}