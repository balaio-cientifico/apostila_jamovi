\chapter{Análise de Dados Categóricos}

A análise de dados categóricos é uma área da estatística que lida com variáveis qualitativas, também conhecidas como variáveis categóricas. Essas variáveis representam características ou atributos que podem ser classificados em categorias ou grupos distintos, mas não podem ser quantificados ou medidos de forma contínua.

Na análise de dados categóricos, o objetivo principal é compreender a distribuição e a associação entre as diferentes categorias, bem como realizar inferências estatísticas a partir desses dados. Isso envolve o uso de técnicas estatísticas específicas que são adequadas para lidar com variáveis categóricas, ao contrário das técnicas usadas para variáveis numéricas contínuas.

Uma das principais ferramentas na análise de dados categóricos é a tabela de contingência, que apresenta a frequência ou a proporção de ocorrência de cada combinação de categorias. A partir dessa tabela, é possível calcular medidas como a frequência relativa, proporções, odds ratio e qui-quadrado, que ajudam a descrever e analisar os padrões nos dados.

Além disso, a análise de dados categóricos inclui técnicas estatísticas como o teste qui-quadrado, que verifica se existe uma associação significativa entre duas ou mais variáveis categóricas. Esse teste compara as frequências observadas com as frequências esperadas, sob a hipótese nula de independência entre as variáveis. Se o valor-p resultante for menor que um nível de significância predefinido, podemos rejeitar a hipótese nula e concluir que há uma associação entre as variáveis.

Outra técnica comum na análise de dados categóricos é a regressão logística, que permite modelar e prever uma variável categórica dependente com base em uma ou mais variáveis independentes. A regressão logística é uma extensão da regressão linear e é especialmente útil quando a variável de interesse é binária (duas categorias), como sim/não, sucesso/fracasso, presente/ausente.

A análise de dados categóricos tem aplicações em diversas áreas, como pesquisas de opinião, epidemiologia, psicologia, marketing e ciências sociais. Ela ajuda a entender a relação entre características categóricas, identificar fatores associados e fazer previsões sobre o comportamento ou a ocorrência de eventos.