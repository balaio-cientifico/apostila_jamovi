\chapter{Testes de Hipóteses}

Os testes de hipóteses são ferramentas fundamentais na área da estatística para auxiliar na tomada de decisões baseadas em evidências empíricas. Esses testes permitem que os pesquisadores avaliem se uma determinada afirmação sobre uma população, chamada de hipótese, é consistente com os dados observados \parencite{Triola2017}.

Em um teste de hipóteses, temos duas hipóteses a serem consideradas: a hipótese nula (H0) e a hipótese alternativa (H1). A hipótese nula é a afirmação que queremos testar, enquanto a hipótese alternativa é a afirmação oposta à hipótese nula \parencite{Bussab2017}.

O procedimento envolve coletar dados da população em estudo e realizar cálculos estatísticos com base nessas informações. O objetivo é obter um valor chamado estatística de teste, que fornece uma medida da força das evidências em relação à hipótese nula. Com base nesse valor, é possível calcular o valor p, que representa a probabilidade de obter uma estatística de teste igual ou mais extrema do que a observada, assumindo que a hipótese nula seja verdadeira \parencite{Magalhaes2015}.

Com base no valor p, os pesquisadores podem tomar uma decisão estatística. Se o valor p for menor do que um nível de significância predefinido, geralmente 0,05, rejeita-se a hipótese nula em favor da hipótese alternativa, indicando que os dados fornecem evidências suficientes para afirmar que a hipótese alternativa é verdadeira. Caso contrário, se o valor p for maior do que o nível de significância, não há evidências suficientes para rejeitar a hipótese nula.

Os testes de hipóteses desempenham um papel importante na pesquisa científica, permitindo que os pesquisadores realizem inferências estatísticas sobre as características de uma população com base em uma amostra limitada de dados. Esses testes fornecem uma estrutura rigorosa para a análise e interpretação dos resultados, contribuindo para a objetividade e a robustez das conclusões estatísticas.

\section{Teste de Hipóteses Unicaudal}