\chapter{Manipulação de Dados no Jamovi}

O foco deste capítulo é a manipulação de dados, uma habilidade essencial para qualquer pessoa que esteja interessada em análise de dados. Vamos explorar as diversas funções e recursos do Jamovi para tratar, organizar e manipular conjuntos de dados. Se você já se perguntou como criar e gerenciar categorias, transformar dados ou manipular variáveis no Jamovi, este capítulo irá te guiar através desses processos passo a passo.

No capítulo anterior você deu os primeiros passos e apresentamos a você a interface do Jamovi para manipulação de dados. Nesse capítulo discutiremos o fluxo de trabalho ideal para o tratamento de um conjunto de dados. O objetivo é garantir que você tenha uma compreensão sólida das ferramentas disponíveis e de como usá-las eficientemente.

Em seguida, abordaremos como criar novas categorias a partir de variáveis existentes. Isso pode ser útil em uma variedade de contextos, como quando você deseja agrupar respostas de pesquisas ou classificar dados em grupos específicos. Também ensinaremos a transformar variáveis, permitindo que você mude o formato dos seus dados de uma maneira que melhor atenda às suas necessidades analíticas.

Finalmente, traremos exemplos práticos para aplicar o conhecimento adquirido. O intuito é promover a familiarização com as ferramentas do software e reforçar a compreensão das funcionalidades abordadas.

Lembre-se, a manipulação eficaz dos dados é a base de qualquer análise de qualidade. Por isso, este capítulo desempenha um papel fundamental no seu aprendizado sobre o uso do Jamovi. Esperamos que ao final desta etapa, você se sinta confiante para manipular conjuntos de dados e prepará-los para a análise de uma maneira eficiente e eficaz.