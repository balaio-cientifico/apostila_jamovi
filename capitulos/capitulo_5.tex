\chapter{Análise de Variância (ANOVA)}

A Análise de Variância (ANOVA) é uma técnica estatística utilizada para comparar as médias de três ou mais grupos independentes. Ela permite determinar se as diferenças observadas entre as médias dos grupos são estatisticamente significativas ou se podem ser atribuídas ao acaso \parencite{Triola2017}.

A ANOVA baseia-se na decomposição da variabilidade total dos dados em duas componentes: a variabilidade entre os grupos e a variabilidade dentro dos grupos. A ideia fundamental por trás da ANOVA é que, se as médias dos grupos são iguais e as variabilidades dentro dos grupos são semelhantes, então qualquer diferença observada entre as médias pode ser atribuída ao acaso. Por outro lado, se as médias dos grupos são diferentes e/ou as variabilidades dentro dos grupos são grandes em relação à variabilidade entre os grupos, então é mais provável que as diferenças observadas sejam estatisticamente significativas.

Na ANOVA, a hipótese nula assume que não há diferenças entre as médias dos grupos, enquanto a hipótese alternativa considera que pelo menos uma das médias é diferente das demais. O teste estatístico da ANOVA utiliza a estatística F, que compara a variabilidade entre os grupos com a variabilidade dentro dos grupos. Se a estatística F é grande o suficiente para rejeitar a hipótese nula, isso indica que pelo menos uma das médias é estatisticamente diferente das demais.

Existem diferentes tipos de ANOVA, dependendo do design do estudo e do número de fatores considerados. A ANOVA de um fator compara as médias de três ou mais grupos independentes, enquanto a ANOVA de dois fatores analisa o efeito de duas variáveis independentes nos grupos. Além disso, a ANOVA pode ser aplicada em diferentes contextos, como experimentos controlados, estudos observacionais ou análise de dados de pesquisas.

A ANOVA é amplamente utilizada em diversas áreas, como ciências sociais, ciências da saúde, ciências naturais e engenharia. Ela fornece uma abordagem estatística robusta para a comparação de médias de vários grupos, permitindo que os pesquisadores determinem se as diferenças observadas são estatisticamente significativas. A ANOVA também permite realizar análises post hoc para identificar quais grupos diferem significativamente entre si, contribuindo para uma compreensão mais aprofundada dos padrões e relações presentes nos dados.