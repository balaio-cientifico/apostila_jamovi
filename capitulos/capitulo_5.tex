\chapter{Regressão Linear}

A regressão linear é uma técnica estatística que estuda e modela a relação entre uma variável dependente (ou resposta) e uma ou mais variáveis independentes (ou preditoras). É amplamente utilizada para prever ou estimar o valor de uma variável dependente com base nos valores conhecidos ou observados das variáveis independentes.

Na regressão linear, a relação entre as variáveis é representada por uma equação linear, que é uma linha reta no caso de uma regressão linear simples com uma variável independente, ou um hiperplano em regressões lineares múltiplas com mais de uma variável independente. A equação linear estima a relação entre as variáveis, permitindo fazer previsões ou inferências sobre o comportamento da variável dependente em função das variáveis independentes.

O objetivo da regressão linear é encontrar os coeficientes da equação linear que melhor se ajustem aos dados. Isso é feito por meio de técnicas de otimização que minimizam a diferença entre os valores previstos pela equação linear e os valores observados dos dados. O coeficiente de inclinação (ou declive) da linha ou hiperplano indica a taxa de mudança da variável dependente para cada unidade de mudança nas variáveis independentes, enquanto o coeficiente de interceptação representa o valor estimado da variável dependente quando todas as variáveis independentes são iguais a zero.

A regressão linear é usada em uma ampla gama de aplicações, desde previsão de vendas e análise de mercado até estudos científicos e análise de dados em ciências sociais. Ela permite examinar e quantificar as relações entre variáveis, identificar tendências e padrões nos dados, e fornecer insights úteis para tomadas de decisão e planejamento.

É importante ressaltar que a regressão linear pressupõe uma relação linear entre as variáveis, e sua eficácia depende da validade dessas suposições. Além disso, existem outras variantes da regressão linear, como regressão linear múltipla, regressão linear ponderada e regressão linear não linear, que permitem lidar com casos mais complexos e não lineares.